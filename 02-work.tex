\cvsection{EXPERIÊNCIA DE TRABALHO}

\cvevent
    {2020 - agora}
    {Fullstack Web Developer}
    {Pluralsight}
    {Atualmente, trabalho como Fullstack Web Developer na Pluralsight. Dentro das minhas atribuições
    ajudo a entregar o backend e frontend de várias aplicações dentro do Flow, produto voltado para coleta e visualização de
    métricas de desenvolvimento.}
    {}
    {\cvlist {
        \item Django + React.js
        \item PostgreSQL + Kafka
        \item Kubernetes + Gitops + Docker
        \item JIRA as the issue tracking/SCRUM tool
    }}
    {}

\cvevent
    {2015 - 2020}
    {CTO e Cofundador}
    {Intelivix}
    {Trabalhei como CTO (diretor de tecnologia) de uma law tech chamada Intelivix. Meu papel nos últimos 5 anos foi focado no gerenciamento de equipes de software (web, engenharia de dados e raspagem de dados), definição da direção de tecnologia da empresa, cuidar de orçamento de produtos e a participação em programas de pesquisa e inovação. Somos investidos pela FINEP e pela FACEPE, agências brasileiras de apoio à ciência e a startups.}
    {}
    {\cvlist {
        \item Django + Backbone.js (Javascript) + HTML5 + Bootstrap Framework (CSS)
        \item Jupyter + Scrapy + Pandas + ScikitLearn + Luigi Framework
        \item PostgreSQL + ElasticSearch + MongoDB
        \item Ansible + Docker for deployments
        \item Amazon Web Services as Cloud Provider
        \item JIRA as the issue tracking/SCRUM tool
    }}
    {}
\cvevent
    {2013 - 2014}
    {Desenvolvedor Web}
    {Claria Seguros}
    {A Claria é uma corretora de seguros. Meu papel foi desenvolver o frontend (Javascript básico / HTML e CSS) e o backend (Django) de uma aplicação web para a venda de seguros de carros. Também trabalhei em códigos de automação para raspar sites de companhias de seguros, o que permitia que os clientes pudessem encontrar facilmente as melhores e mais recentes ofertas disponíveis. Usei a AWS como nosso provedor de máquinas e o Google Adwords e o Google Analytics para aquisição de leads e estratégias de marketing.}
    {}
    {\cvlist {
        \item Django + Scrapy + Celery (Python)
        \item Backbone.js (Javascript) + HTML5 + Bootstrap Framework (CSS)
        \item PostgreSQL
        \item Amazon Web Services as Cloud Provider
        \item JIRA as the issue tracking/SCRUM tool
    }}
    {}
\cvevent
    {2011 - 2013}
    {Desenvolvedor e Cofundador}
    {Widevis Tecnologia em Des. de Software LTDA}
    {Após a conclusão do meu mestrado, abri uma empresa chamada Widevis. Ela foi selecionada para um programa de pré-incubação do Porto Digital. Dentro do programa tivemos acesso a aulas sobre empreendedorismo e consultorias especializadas em negócios de tecnologia. Nosso objetivo era desenvolver uma rede social para arrecadar dinheiro para os times de futebol brasileiros em troca do uso de seus estádios, interações entre jogadores, mercado e transparência orçamentária.}
    {}
    {\cvlist {
        \item Django + Scrapy (Python)
        \item PostgreSQL
    }}
    {}
\cvevent
    {2008 - 2009}
    {Estágio}
    {5IT Des. de Sistemas de Informação LTDA}
    {Durante minha graduação, iniciei um programa de estágio chamada BITEC promovido pelo Instituto Euvaldo Lodi (IEL). Este programa era ao mesmo tempo um estágio em desenvolvimento de software e uma bolsa de iniciação científica. Neste estágio, trabalhei com uma aplicação Web ColdFusion conectada a um banco de dados MySQL.}
    {}
    {\cvlist {
        \item Adobe ColdFusion
        \item MySQL
    }}
    {}
