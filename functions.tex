% returns minipage width minus two times \fboxsep
% to keep padding included in width calculations
% can also be used for other boxes / environments
\newcommand{\mpwidth}{\linewidth-\fboxsep-\fboxsep}


%============================================================================%
%
%   CV COMMANDS
%
%============================================================================%

%----------------------------------------------------------------------------------------
%    CV LIST
%----------------------------------------------------------------------------------------

% renders a standard latex list but abstracts away the environment definition (begin/end)
\newcommand{\cvlist}[1] {
    \begin{itemize}{#1}\end{itemize}
}

%----------------------------------------------------------------------------------------
%    CV TEXT
%----------------------------------------------------------------------------------------

% base class to wrap any text based stuff here. Renders like a paragraph.
% Allows complex commands to be passed, too.
% param 1: *any
\newcommand{\cvtext}[1] {
    \begin{tabular*}{1\mpwidth}{p{0.98\mpwidth}}
        \parbox{1\mpwidth}{#1}
    \end{tabular*}
}

%----------------------------------------------------------------------------------------
%   CV SECTION
%----------------------------------------------------------------------------------------

% Renders a a CV section headline with a nice underline in main color.
% param 1: section title
\newcommand{\cvsection}[1] {
    \vspace{14pt}
    \cvtext{
        \textbf{\LARGE{\textcolor{darkcol}{\uppercase{#1}}}}\\[-4pt]
        \textcolor{maincol}{ \rule{0.1\textwidth}{2pt} } \\
    }
}

%----------------------------------------------------------------------------------------
%   META SKILL
%----------------------------------------------------------------------------------------

% Renders a progress-bar to indicate a certain skill in percent.
% param 1: name of the skill / tech / etc.
% param 2: level (for example in years)
% param 3: percent, values range from 0 to 1
\newcommand{\cvskill}[3] {
    \begin{tabular*}{1\mpwidth}{p{0.72\mpwidth}  r}
        \textcolor{black}{\textbf{#1}} & \textcolor{maincol}{#2}\\
    \end{tabular*}%

    \hspace{4pt}
    \begin{tikzpicture}[scale=1,rounded corners=2pt,very thin]
        \fill [lightcol] (0,0) rectangle (1\mpwidth, 0.15);
        \fill [maincol] (0,0) rectangle (#3\mpwidth, 0.15);
    \end{tikzpicture}%
}

\newcommand{\cvlanguage}[3] {
    \begin{tabular*}{1\mpwidth}{p{0.72\mpwidth}  r}
        \textcolor{black}{\textbf{#1}}
    \end{tabular*}%

    \begin{tabular*}{1\mpwidth}{p{0.72\mpwidth}  r}
        \textcolor{maincol}{#2}
    \end{tabular*}%

    \begin{tikzpicture}[scale=1,rounded corners=2pt,very thin]
        \fill [lightcol] (0,0) rectangle (1\mpwidth, 0.15);
        \fill [maincol] (0,0) rectangle (#3\mpwidth, 0.15);
    \end{tikzpicture}%
}


%----------------------------------------------------------------------------------------
%    CV EVENT
%----------------------------------------------------------------------------------------

% Renders a table and a paragraph (cvtext) wrapped in a parbox (to ensure minimum content
% is glued together when a pagebreak appears).
% Additional Information can be passed in text or list form (or other environments).
% the work you did
% param 1: time-frame i.e. Sep 14 - Jan 15 etc.
% param 2:   event name (job position etc.)
% param 3: Customer, Employer, Industry
% param 4: Short description
% param 5: work done (optional)
% param 6: technologies include (optional)
% param 7: achievements (optional)
\newcommand{\cvevent}[7] {

    % we wrap this part in a parbox, so title and description are not separated on a pagebreak
    % if you need more control on page breaks, remove the parbox
    \parbox{\mpwidth}{
        \begin{tabular*}{1\mpwidth}{p{0.72\mpwidth}  r}
            \textcolor{black}{\textbf{#2}} & \colorbox{maincol}{\makebox[0.25\mpwidth]{\textcolor{white}{#1}}} \\
            \textcolor{maincol}{\textbf{#3}} & \\
        \end{tabular*}\\[8pt]

        \ifthenelse{\isempty{#4}}{}{
            \cvtext{#4}\\
        }
    }

    \ifthenelse{\isempty{#5}}{}{
        \vspace{9pt}
        {#5}
    }

    \ifthenelse{\isempty{#6}}{}{
        \vspace{9pt}
        \cvtext{\textbf{Technologies include:}}\\
        {#6}
    }

    \ifthenelse{\isempty{#7}}{}{
        \vspace{9pt}
        \cvtext{\textbf{Achievements include:}}\\
        {#7}
    }
    \vspace{14pt}
}

%----------------------------------------------------------------------------------------
%    CV META EVENT
%----------------------------------------------------------------------------------------

% Renders a CV event on the sidebar
% param 1: title
% param 2: subtitle (optional)
% param 3: customer, employer, etc,. (optional)
% param 4: info text (optional)
\newcommand{\cvmetaevent}[4] {
    \textcolor{maincol} {\cvtext{\textbf{\begin{flushleft}#1\end{flushleft}}}}

    \ifthenelse{\isempty{#2}}{}{
    \textcolor{darkcol} {\cvtext{\textbf{#2}} }
    }

    \ifthenelse{\isempty{#3}}{}{
        \cvtext{{ \textcolor{darkcol} {#3} }}\\
    }

    \ifthenelse{\isempty{#4}}{}{
        \cvtext{#4}\\[14pt]
    }
}

%---------------------------------------------------------------------------------------
%   HEADER
%----------------------------------------------------------------------------------------
\definecolor{headerblue}{HTML}{2EB6E1}%
\definecolor{headerfontbox}{HTML}{000000}
\definecolor{headerfontboxfont}{HTML}{FFFFFF}

\newcommand{\bg}[3]{\colorbox{#1}{\bfseries\color{#2}#3}}
\newcommand{\bgupper}[3]{\colorbox{#1}{\color{#2}\huge\bfseries\MakeUppercase{#3}}}

\newcommand{\header}[7]{
\tikz[remember picture,overlay] {%
\node[rectangle, fill=#5, anchor=north, minimum width=\paperwidth, minimum height=5cm](header) at (current page.north){};%
\node[left=#7 of header.north, anchor=east](name) at (header.east) {\fontfamily{\sfdefault}\selectfont #2};%
\node[anchor=south east](degree) at (name.north east) {\fontfamily{\sfdefault}\selectfont #1};%
\node[anchor=north east](descr) at (name.south east) {\fontfamily{\sfdefault}\selectfont #3};%
\node[right=#6 of header.west, anchor=west](picture) at (header.west) {};%
\draw[path picture={\node[anchor=center, fill=#5] at (path picture bounding box.center){\includegraphics[height=4.3cm,angle=90,origin=c]{#4}} ;}] (picture) circle (2) ;}%
\vspace{1.5cm}%
}

%---------------------------------------------------------------------------------------
%   QR CODE
%----------------------------------------------------------------------------------------

% Renders a qrcode image (centered, relative to the parentwidth)
% param 1: percent width, from 0 to 1
\newcommand{\cvqrcode}[1] {
    \begin{center}
        \includegraphics[width={#1}\mpwidth]{qr-code.pdf}
    \end{center}
}


\newcommand{\cvqrcodevcard}[1] {
    \begin{center}
        \includegraphics[width={#1}\mpwidth]{qr-code-vcard.pdf}
    \end{center}
}
